\documentclass[12pt,a4paper]{article} %Formato de plantilla a utilizar
\usepackage[utf8]{inputenc}  %paquete para trabajar en español
\usepackage[spanish]{babel} %Paquete para trabajar en español
\usepackage{graphicx} %Para insertar imagenes
\usepackage[left=2.54cm, right=2.54cm, top=2.54cm]{geometry} %Margenes
\usepackage{fancyhdr} %Define el estilo de la página y cabecera
\usepackage{parskip} %Arreglo de la tabulacion
\usepackage{amsmath}
\usepackage[hidelinks]{hyperref}
\usepackage[usenames, dvipsnames]{xcolor}
\usepackage{xspace}
\usepackage{subfig}
\usepackage{tikz}
\usepackage{circuitikz}
\usepackage[T1]{fontenc}
\usepackage{adjustbox}
\usepackage{amssymb}
\usepackage{times}
\usepackage{grffile}
\usepackage{etoolbox}
\usepackage{pgfplots}
\usepackage{graphicx}
\usepackage{adjustbox}
\usepackage{amssymb}
\usepackage{gensymb}
\usepackage{listings}
\usepackage{xcolor}
\usepackage{float}
\usepackage{pgf}
\usepackage{hyperref}
\usepackage{titlesec}
\usepackage{booktabs}
\usetikzlibrary{decorations.pathreplacing}
\usetikzlibrary{babel}

%&Colores para introducir el codigo de aklgun lenguaje de programacion
\definecolor{codegreen}{rgb}{0,0.6,0}
\definecolor{codegray}{rgb}{0.5,0.5,0.5}
\definecolor{codepurple}{rgb}{0.58,0,0.82}
\definecolor{backcolor}{rgb}{0.95,0.95,0.92}

\lstdefinestyle{mystyle}{
	backgroundcolor=\color{backcolor},
	commentstyle=\color{codegreen},
	keywordstyle=\color{magenta},
	numberstyle=\tiny\color{codegray},
	stringstyle=\color{codepurple},
	basicstyle=\ttfamily\footnotesize,
	breakatwhitespace=false,
	breaklines=true,
	captionpos=b,
	keepspaces=true,
	numbers=left,
	showspaces=false,
	showstringspaces=false,
	showtabs=false,
	tabsize=2
}

\titleformat{\paragraph}
{\normalfont\normalsize\bfseries}{\theparagraph}{1em}{}
\titlespacing*{\paragraph}{0pt}{3.25ex plus 1ex minus .2ex}{1.5ex plus .2ex}

\setcounter{secnumdepth}{4} % Configura la profundidad de numeración de secciones
\setcounter{tocdepth}{4} % Configura la profundidad de inclusión en el índice

%Declaración de variables
\newcommand{\logouni}{Imagenes/Logo.png}
\newcommand{\startDate}{29 de Octubre del 2023}
\renewcommand{\contentsname}{}

%Cabecera
\pagestyle{fancy}
\setlength{\headheight}{50.2pt}
\fancyhf{}
\chead{\includegraphics[width=2cm]{\logouni}}
\lhead{CT3233 - Sistemas de potencia I\\ Ing. Luis Andrade}
\rhead{Tarea \#2\\ Semana 8}
%Pie de página
\lfoot{Tema \#2}
\cfoot{Página\ \thepage}
\rfoot{Métodos Numéricos}
\renewcommand{\footrulewidth}{0.08pt}
\setcounter{page}{2}
%Comienzo del documento
\begin{document}

% Inicio de la portada
	\begin{titlepage}
	%parte superior
		\centering
		\includegraphics[width=0.3\textwidth]{\logouni}\par\vspace{0.2cm}
		{\scshape\LARGE UNIVERSIDAD SIMÓN BOLIVAR}\vspace{0.2cm}\\
		{\scshape \large DEPARTAMENTO DE CONVERSIÓN Y TRANSPORTE DE ENERGÍA}\vspace{0.2cm}\\
		{\large CT-3233 - SISTEMAS DE POTENCIA I}\par
		{\large SECCIÓN 1}\par
		\vspace{5cm}
	%Titulo del trabajo
		\centering
		{\scshape \LARGE \textbf{TAREA\#2:}\par \vspace{0.2cm} ECUACIONES NO LINEALES POR JACOBI, GS Y NR}\par
		\vspace{5cm}
	% nombre del estudiante
		\begin{minipage}{0.25\textwidth}
			{\large \textbf{Autor:}}\\
			{\large Leonardo Madriz \\ Carnet:  20-10399}
		\end{minipage}
		\hfill
		\begin{minipage}{0.25\textwidth}
			{\large \textbf{Profesor:}}\\
			{\large Ing. Luis Andrade}
		\end{minipage}
	\end{titlepage}
% Final de la portada
	\clearpage
%----------------------------------------------------------------------------------------------------------------------------------------------------------------------
  	\tableofcontents
	\clearpage
%----------------------------------------------------------------------------------------------------------------------------------------------------------------------
	 \section{INSTRUCCIONES}
Realice los problemas Propuestos del 3.2  del documento P2.pdf

Entregue la tarea en formato PDF especficando:

\begin{itemize}
    \item  Expresión y primera iteración. No olvide expresar los Jacobianos para dicha expresión cuando aplique NR.
    \item Tabla de iteraciones soluciones y con errores.
    \item Análisis de resultados.
    \item Scripts de apoyo en .m desarrollados en Octave.
    \item Entrega en formato pdf de la tarea.
\end{itemize}

%----------------------------------------------------------------------------------------------------------------------------------------------------------------------
	\section{CODIGO DE PYTHON}
    \subsection{Método de Jacobi}
 \begin{lstlisting}[language=Python,style=mystyle]
import sympy as sp
import pandas as pd

def run():
#Declaracion de variables y funciones
    x,y,z = sp.symbols('x y z') #Variables del metodo
    g1 = sp.sympify()           #x=g1(x,y)
    g2 = sp.sympify(18-x**2)    #y=g2(x,y)

#Algoritmo de Jacobi
    #Condiciones iniciales
    i = 0; x0 = 0; y0 = 0
    #Lista de la salida
    iteraciones = []; x_values = []; y_values = []; error_values = []
    #Parte iterativa
    while True:
        #Calculo de los nuevos valores
        x_iter = g1.subs({x: x0, y: y0})
        y_iter = g2.subs({x: x0, y: y0})
        error = max(abs(x_iter - x0),abs(y_iter - y0))
        #Agregar valores a los arrays de salida
        iteraciones.append(i)
        x_values.append(float("{:.4f}".format(x0)))
        y_values.append(float("{:.4f}".format(y0)))
        error_values.append(float("{:.4f}".format(float(error))))
        #Condicion para reiterar
        if (error < 10e-4) or (i >50):
            break
        else:
           #Actualizacion de variables
           i += 1
           x0 = x_iter
           y0 = y_iter

#Guardado de los valores en latex
    df = pd.DataFrame({
        'ITERACIONES': iteraciones,
        'X' : x_values,
        'Y' : y_values,
        'ERROR': error_values
    })

    latex_table = df.to_latex(index=False, caption="Metodo de Jacobi", label="Jacobi Table")
    print(latex_table)

if __name__ == "__main__":
    run()
	\end{lstlisting}

	\subsection{Método de Gauss-Seidel}

%----------------------------------------------------------------------------------------------------------------------------------------------------------------------
	\include{problema1}
%----------------------------------------------------------------------------------------------------------------------------------------------------------------------
    \include{problema2}
%----------------------------------------------------------------------------------------------------------------------------------------------------------------------
    \include{problema3}
%----------------------------------------------------------------------------------------------------------------------------------------------------------------------
    \include{problema4}
\end{document}

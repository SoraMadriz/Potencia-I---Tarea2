\section{CODIGO DE PYTHON}
    \subsection{Método de Jacobi}
 \begin{lstlisting}[language=Python,style=mystyle]
import sympy as sp
import pandas as pd

def run():
#Declaracion de variables y funciones
    x,y,z = sp.symbols('x y z') #Variables del metodo
    g1 = sp.sympify()           #x=g1(x,y)
    g2 = sp.sympify(18-x**2)    #y=g2(x,y)

#Algoritmo de Jacobi
    #Condiciones iniciales
    i = 0; x0 = 0; y0 = 0
    #Lista de la salida
    iteraciones = []; x_values = []; y_values = []; error_values = []
    #Parte iterativa
    while True:
        #Calculo de los nuevos valores
        x_iter = g1.subs({x: x0, y: y0})
        y_iter = g2.subs({x: x0, y: y0})
        error = max(abs(x_iter - x0),abs(y_iter - y0))
        #Agregar valores a los arrays de salida
        iteraciones.append(i)
        x_values.append(float("{:.4f}".format(x0)))
        y_values.append(float("{:.4f}".format(y0)))
        error_values.append(float("{:.4f}".format(float(error))))
        #Condicion para reiterar
        if (error < 10e-4) or (i >50):
            break
        else:
           #Actualizacion de variables
           i += 1
           x0 = x_iter
           y0 = y_iter

#Guardado de los valores en latex
    df = pd.DataFrame({
        'ITERACIONES': iteraciones,
        'X' : x_values,
        'Y' : y_values,
        'ERROR': error_values
    })

    latex_table = df.to_latex(index=False, caption="Metodo de Jacobi", label="Jacobi Table")
    print(latex_table)

if __name__ == "__main__":
    run()
	\end{lstlisting}

	\subsection{Método de Gauss-Seidel}
